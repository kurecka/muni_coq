\documentclass[12pt]{report}
\usepackage[]{inputenc}
\usepackage[T1]{fontenc}
\usepackage{fullpage}
\usepackage{coqdoc}
\usepackage{amsmath,amssymb}
\usepackage{url}
\begin{document}
%%%%%%%%%%%%%%%%%%%%%%%%%%%%%%%%%%%%%%%%%%%%%%%%%%%%%%%%%%%%%%%%%
%% This file has been automatically generated with the command
%% coqdoc -q -g --latex -o Induction.tex Induction.v 
%%%%%%%%%%%%%%%%%%%%%%%%%%%%%%%%%%%%%%%%%%%%%%%%%%%%%%%%%%%%%%%%%
\coqlibrary{Induction}{Library }{Induction}

\begin{coqdoccode}
\end{coqdoccode}
\section{Induction: Proof by Induction}



 Before getting started, we need to import all of our
    definitions from the previous chapter: \begin{coqdoccode}
\coqdocemptyline
\coqdocemptyline
\end{coqdoccode}
For the \coqdockw{Require} \coqdockw{Export} to work, you first need to use
    \coqdocvar{coqc} to compile \coqdocvar{Basics.v} into \coqdocvar{Basics.vo}.  This is like
    making a .\coqdocvar{class} file from a .\coqdocvar{java} file, or a .\coqdocvar{o} file from a
    .\coqdocvar{c} file.  There are two ways to do it:



\begin{itemize}
\item  In CoqIDE:


         Open \coqdocvar{Basics.v}.  In the ``Compile'' menu, click on ``Compile
         Buffer''.



\item  From the command line: Either


         \coqdocvar{make} \coqdocvar{Basics.vo}


       (assuming you've downloaded the whole LF directory and have a
       working \coqdocvar{make} command) or


         \coqdocvar{coqc} \coqdocvar{Basics.v}


       (which should work regardless).

\end{itemize}


    If you have trouble (e.g., if you get complaints about missing
    identifiers later in the file), it may be because the ``load path''
    for Coq is not set up correctly.  The \coqdockw{Print} \coqdocvar{LoadPath}. command may
    be helpful in sorting out such issues.


    In particular, if you see a message like


        \coqdocvar{Compiled} \coqdocvar{library} \coqdocvar{Foo} \coqdocvar{makes} \coqdocvar{inconsistent} \coqdocvar{assumptions} \coqdocvar{over}
        \coqdocvar{library} \coqdocvar{Coq.Init.Bar}


    you should check whether you have multiple installations of Coq on
    your machine.  If so, it may be that commands (like \coqdocvar{coqc}) that
    you execute in a terminal window are getting a different version of
    Coq than commands executed by Proof General or CoqIDE.


    One more tip for CoqIDE users: If you see messages like \coqdocvar{Error}:
    \coqdocvar{Unable} \coqdocvar{to} \coqdocvar{locate} \coqdocvar{library} \coqdocvar{Basics}, a likely reason is
    inconsistencies between compiling things \textit{within CoqIDE} vs \textit{using
    coqc} from the command line.  This typically happens when there are
    two incompatible versions of \coqdocvar{coqc} installed on your system (one
    associated with CoqIDE, and one associated with \coqdocvar{coqc} from the
    terminal).  The workaround for this situation is compiling using
    CoqIDE only (i.e. choosing ``make'' from the menu), and avoiding
    using \coqdocvar{coqc} directly at all. \begin{coqdoccode}
\coqdocemptyline
\end{coqdoccode}
\section{Proof by Induction}



 We proved in the last chapter that 0 is a neutral element
    for + on the left, using an easy argument based on
    simplification.  We also observed that proving the fact that it is
    also a neutral element on the \textit{right}... \begin{coqdoccode}
\coqdocemptyline
\coqdocnoindent
\coqdockw{Theorem} \coqdocvar{plus\_n\_O\_firsttry} : \coqdockw{\ensuremath{\forall}} \coqdocvar{n}:\coqdocvar{nat},\coqdoceol
\coqdocindent{1.00em}
\coqdocvar{n} = \coqdocvar{n} + 0.\coqdoceol
\end{coqdoccode}
... can't be done in the same simple way.  Just applying
  \coqdoctac{reflexivity} doesn't work, since the \coqdocvar{n} in \coqdocvar{n} + 0 is an arbitrary
  unknown number, so the \coqdockw{match} in the definition of + can't be
  simplified.  \begin{coqdoccode}
\coqdocemptyline
\end{coqdoccode}
And reasoning by cases using \coqdoctac{destruct} \coqdocvar{n} doesn't get us much
    further: the branch of the case analysis where we assume \coqdocvar{n} = 0
    goes through fine, but in the branch where \coqdocvar{n} = \coqdocvar{S} \coqdocvar{n'} for some \coqdocvar{n'} we
    get stuck in exactly the same way. \begin{coqdoccode}
\coqdocemptyline
\coqdocnoindent
\coqdockw{Theorem} \coqdocvar{plus\_n\_O\_secondtry} : \coqdockw{\ensuremath{\forall}} \coqdocvar{n}:\coqdocvar{nat},\coqdoceol
\coqdocindent{1.00em}
\coqdocvar{n} = \coqdocvar{n} + 0.\coqdoceol
\coqdocemptyline
\end{coqdoccode}
We could use \coqdoctac{destruct} \coqdocvar{n'} to get one step further, but,
    since \coqdocvar{n} can be arbitrarily large, if we just go on like this
    we'll never finish. 

 To prove interesting facts about numbers, lists, and other
    inductively defined sets, we usually need a more powerful
    reasoning principle: \textit{induction}.


    Recall (from high school, a discrete math course, etc.) the
    \textit{principle of induction over natural numbers}: If \coqdocvar{P}(\coqdocvar{n}) is some
    proposition involving a natural number \coqdocvar{n} and we want to show
    that \coqdocvar{P} holds for all numbers \coqdocvar{n}, we can reason like this:

\begin{itemize}
\item  show that \coqdocvar{P}(\coqdocvar{O}) holds;

\item  show that, for any \coqdocvar{n'}, if \coqdocvar{P}(\coqdocvar{n'}) holds, then so does
           \coqdocvar{P}(\coqdocvar{S} \coqdocvar{n'});

\item  conclude that \coqdocvar{P}(\coqdocvar{n}) holds for all \coqdocvar{n}.

\end{itemize}


    In Coq, the steps are the same: we begin with the goal of proving
    \coqdocvar{P}(\coqdocvar{n}) for all \coqdocvar{n} and break it down (by applying the \coqdoctac{induction}
    tactic) into two separate subgoals: one where we must show \coqdocvar{P}(\coqdocvar{O})
    and another where we must show \coqdocvar{P}(\coqdocvar{n'}) \ensuremath{\rightarrow} \coqdocvar{P}(\coqdocvar{S} \coqdocvar{n'}).  Here's how
    this works for the theorem at hand: \begin{coqdoccode}
\coqdocemptyline
\coqdocnoindent
\coqdockw{Theorem} \coqdocvar{plus\_n\_O} : \coqdockw{\ensuremath{\forall}} \coqdocvar{n}:\coqdocvar{nat}, \coqdocvar{n} = \coqdocvar{n} + 0.\coqdoceol
\coqdocemptyline
\end{coqdoccode}
Like \coqdoctac{destruct}, the \coqdoctac{induction} tactic takes an \coqdockw{as}...
    clause that specifies the names of the variables to be introduced
    in the subgoals.  Since there are two subgoals, the \coqdockw{as}... clause
    has two parts, separated by \ensuremath{|}.  (Strictly speaking, we can omit
    the \coqdockw{as}... clause and Coq will choose names for us.  In practice,
    this is a bad idea, as Coq's automatic choices tend to be
    confusing.)


    In the first subgoal, \coqdocvar{n} is replaced by 0.  No new variables
    are introduced (so the first part of the \coqdockw{as}... is empty), and
    the goal becomes 0 = 0 + 0, which follows by simplification.


    In the second subgoal, \coqdocvar{n} is replaced by \coqdocvar{S} \coqdocvar{n'}, and the
    assumption \coqdocvar{n'} + 0 = \coqdocvar{n'} is added to the context with the name
    \coqdocvar{IHn'} (i.e., the Induction Hypothesis for \coqdocvar{n'}).  These two names
    are specified in the second part of the \coqdockw{as}... clause.  The goal
    in this case becomes \coqdocvar{S} \coqdocvar{n'} = (\coqdocvar{S} \coqdocvar{n'}) + 0, which simplifies to
    \coqdocvar{S} \coqdocvar{n'} = \coqdocvar{S} (\coqdocvar{n'} + 0), which in turn follows from \coqdocvar{IHn'}. \begin{coqdoccode}
\coqdocemptyline
\coqdocnoindent
\coqdockw{Theorem} \coqdocvar{minus\_diag} : \coqdockw{\ensuremath{\forall}} \coqdocvar{n},\coqdoceol
\coqdocindent{1.00em}
\coqdocvar{minus} \coqdocvar{n} \coqdocvar{n} = 0.\coqdoceol
\coqdocemptyline
\end{coqdoccode}
(The use of the \coqdoctac{intros} tactic in these proofs is actually
    redundant.  When applied to a goal that contains quantified
    variables, the \coqdoctac{induction} tactic will automatically move them
    into the context as needed.) 

\paragraph{Exercise: 2 stars, recommended (basic\_induction)}

 Prove the following using induction. You might need previously
    proven results. \begin{coqdoccode}
\coqdocemptyline
\coqdocnoindent
\coqdockw{Theorem} \coqdocvar{mult\_0\_r} : \coqdockw{\ensuremath{\forall}} \coqdocvar{n}:\coqdocvar{nat},\coqdoceol
\coqdocindent{1.00em}
\coqdocvar{n} \ensuremath{\times} 0 = 0.\coqdoceol
 \coqdocemptyline
\coqdocnoindent
\coqdockw{Theorem} \coqdocvar{plus\_n\_Sm} : \coqdockw{\ensuremath{\forall}} \coqdocvar{n} \coqdocvar{m} : \coqdocvar{nat},\coqdoceol
\coqdocindent{1.00em}
\coqdocvar{S} (\coqdocvar{n} + \coqdocvar{m}) = \coqdocvar{n} + (\coqdocvar{S} \coqdocvar{m}).\coqdoceol
 \coqdocemptyline
\coqdocnoindent
\coqdockw{Theorem} \coqdocvar{plus\_comm} : \coqdockw{\ensuremath{\forall}} \coqdocvar{n} \coqdocvar{m} : \coqdocvar{nat},\coqdoceol
\coqdocindent{1.00em}
\coqdocvar{n} + \coqdocvar{m} = \coqdocvar{m} + \coqdocvar{n}.\coqdoceol
 \coqdocemptyline
\coqdocnoindent
\coqdockw{Theorem} \coqdocvar{plus\_assoc} : \coqdockw{\ensuremath{\forall}} \coqdocvar{n} \coqdocvar{m} \coqdocvar{p} : \coqdocvar{nat},\coqdoceol
\coqdocindent{1.00em}
\coqdocvar{n} + (\coqdocvar{m} + \coqdocvar{p}) = (\coqdocvar{n} + \coqdocvar{m}) + \coqdocvar{p}.\coqdoceol
 \end{coqdoccode}
\ensuremath{\Box} 

\paragraph{Exercise: 2 stars (double\_plus)}

 Consider the following function, which doubles its argument: \begin{coqdoccode}
\coqdocemptyline
\coqdocnoindent
\coqdockw{Fixpoint} \coqdocvar{double} (\coqdocvar{n}:\coqdocvar{nat}) :=\coqdoceol
\coqdocindent{1.00em}
\coqdockw{match} \coqdocvar{n} \coqdockw{with}\coqdoceol
\coqdocindent{1.00em}
\ensuremath{|} \coqdocvar{O} \ensuremath{\Rightarrow} \coqdocvar{O}\coqdoceol
\coqdocindent{1.00em}
\ensuremath{|} \coqdocvar{S} \coqdocvar{n'} \ensuremath{\Rightarrow} \coqdocvar{S} (\coqdocvar{S} (\coqdocvar{double} \coqdocvar{n'}))\coqdoceol
\coqdocindent{1.00em}
\coqdockw{end}.\coqdoceol
\coqdocemptyline
\end{coqdoccode}
Use induction to prove this simple fact about \coqdocvar{double}: \begin{coqdoccode}
\coqdocemptyline
\coqdocnoindent
\coqdockw{Lemma} \coqdocvar{double\_plus} : \coqdockw{\ensuremath{\forall}} \coqdocvar{n}, \coqdocvar{double} \coqdocvar{n} = \coqdocvar{n} + \coqdocvar{n} .\coqdoceol
 \end{coqdoccode}
\ensuremath{\Box} 

\paragraph{Exercise: 2 stars, optional (evenb\_S)}

 One inconvenient aspect of our definition of \coqdocvar{evenb} \coqdocvar{n} is the
    recursive call on \coqdocvar{n} - 2. This makes proofs about \coqdocvar{evenb} \coqdocvar{n}
    harder when done by induction on \coqdocvar{n}, since we may need an
    induction hypothesis about \coqdocvar{n} - 2. The following lemma gives an
    alternative characterization of \coqdocvar{evenb} (\coqdocvar{S} \coqdocvar{n}) that works better
    with induction: \begin{coqdoccode}
\coqdocemptyline
\coqdocnoindent
\coqdockw{Theorem} \coqdocvar{evenb\_S} : \coqdockw{\ensuremath{\forall}} \coqdocvar{n} : \coqdocvar{nat},\coqdoceol
\coqdocindent{1.00em}
\coqdocvar{evenb} (\coqdocvar{S} \coqdocvar{n}) = \coqdocvar{negb} (\coqdocvar{evenb} \coqdocvar{n}).\coqdoceol
 \end{coqdoccode}
\ensuremath{\Box} 

\paragraph{Exercise: 1 star (destruct\_induction)}

 Briefly explain the difference between the tactics \coqdoctac{destruct}
    and \coqdoctac{induction}.


\begin{coqdoccode}
\coqdocemptyline
\coqdocnoindent
\coqdockw{Definition} \coqdocvar{manual\_grade\_for\_destruct\_induction} : \coqdocvar{option} (\coqdocvar{prod} \coqdocvar{nat} \coqdocvar{string}) := \coqdocvar{None}.\coqdoceol
\end{coqdoccode}
\ensuremath{\Box} \begin{coqdoccode}
\coqdocemptyline
\end{coqdoccode}
\section{Proofs Within Proofs}



 In Coq, as in informal mathematics, large proofs are often
    broken into a sequence of theorems, with later proofs referring to
    earlier theorems.  But sometimes a proof will require some
    miscellaneous fact that is too trivial and of too little general
    interest to bother giving it its own top-level name.  In such
    cases, it is convenient to be able to simply state and prove the
    needed ``sub-theorem'' right at the point where it is used.  The
    \coqdoctac{assert} tactic allows us to do this.  For example, our earlier
    proof of the \coqdocvar{mult\_0\_plus} theorem referred to a previous theorem
    named \coqdocvar{plus\_O\_n}.  We could instead use \coqdoctac{assert} to state and
    prove \coqdocvar{plus\_O\_n} in-line: \begin{coqdoccode}
\coqdocemptyline
\coqdocnoindent
\coqdockw{Theorem} \coqdocvar{mult\_0\_plus'} : \coqdockw{\ensuremath{\forall}} \coqdocvar{n} \coqdocvar{m} : \coqdocvar{nat},\coqdoceol
\coqdocindent{1.00em}
(0 + \coqdocvar{n}) \ensuremath{\times} \coqdocvar{m} = \coqdocvar{n} \ensuremath{\times} \coqdocvar{m}.\coqdoceol
\coqdocemptyline
\end{coqdoccode}
The \coqdoctac{assert} tactic introduces two sub-goals.  The first is
    the assertion itself; by prefixing it with \coqdocvar{H}: we name the
    assertion \coqdocvar{H}.  (We can also name the assertion with \coqdockw{as} just as
    we did above with \coqdoctac{destruct} and \coqdoctac{induction}, i.e., \coqdoctac{assert} (0 + \coqdocvar{n}
    = \coqdocvar{n}) \coqdockw{as} \coqdocvar{H}.)  Note that we surround the proof of this assertion
    with curly braces \{ ... \}, both for readability and so that,
    when using Coq interactively, we can see more easily when we have
    finished this sub-proof.  The second goal is the same as the one
    at the point where we invoke \coqdoctac{assert} except that, in the context,
    we now have the assumption \coqdocvar{H} that 0 + \coqdocvar{n} = \coqdocvar{n}.  That is,
    \coqdoctac{assert} generates one subgoal where we must prove the asserted
    fact and a second subgoal where we can use the asserted fact to
    make progress on whatever we were trying to prove in the first
    place. 

 Another example of \coqdoctac{assert}... 

 For example, suppose we want to prove that (\coqdocvar{n} + \coqdocvar{m}) + (\coqdocvar{p} + \coqdocvar{q})
    = (\coqdocvar{m} + \coqdocvar{n}) + (\coqdocvar{p} + \coqdocvar{q}). The only difference between the two sides of
    the = is that the arguments \coqdocvar{m} and \coqdocvar{n} to the first inner +
    are swapped, so it seems we should be able to use the
    commutativity of addition (\coqdocvar{plus\_comm}) to rewrite one into the
    other.  However, the \coqdoctac{rewrite} tactic is not very smart about
    \textit{where} it applies the rewrite.  There are three uses of + here,
    and it turns out that doing \coqdoctac{rewrite} \ensuremath{\rightarrow} \coqdocvar{plus\_comm} will affect
    only the \textit{outer} one... \begin{coqdoccode}
\coqdocemptyline
\coqdocnoindent
\coqdockw{Theorem} \coqdocvar{plus\_rearrange\_firsttry} : \coqdockw{\ensuremath{\forall}} \coqdocvar{n} \coqdocvar{m} \coqdocvar{p} \coqdocvar{q} : \coqdocvar{nat},\coqdoceol
\coqdocindent{1.00em}
(\coqdocvar{n} + \coqdocvar{m}) + (\coqdocvar{p} + \coqdocvar{q}) = (\coqdocvar{m} + \coqdocvar{n}) + (\coqdocvar{p} + \coqdocvar{q}).\coqdoceol
\coqdocemptyline
\end{coqdoccode}
To use \coqdocvar{plus\_comm} at the point where we need it, we can introduce
    a local lemma stating that \coqdocvar{n} + \coqdocvar{m} = \coqdocvar{m} + \coqdocvar{n} (for the particular \coqdocvar{m}
    and \coqdocvar{n} that we are talking about here), prove this lemma using
    \coqdocvar{plus\_comm}, and then use it to do the desired rewrite. \begin{coqdoccode}
\coqdocemptyline
\coqdocnoindent
\coqdockw{Theorem} \coqdocvar{plus\_rearrange} : \coqdockw{\ensuremath{\forall}} \coqdocvar{n} \coqdocvar{m} \coqdocvar{p} \coqdocvar{q} : \coqdocvar{nat},\coqdoceol
\coqdocindent{1.00em}
(\coqdocvar{n} + \coqdocvar{m}) + (\coqdocvar{p} + \coqdocvar{q}) = (\coqdocvar{m} + \coqdocvar{n}) + (\coqdocvar{p} + \coqdocvar{q}).\coqdoceol
\coqdocemptyline
\end{coqdoccode}
\section{Formal vs. Informal Proof}



 "\textit{Informal proofs are algorithms; formal proofs are code}.`` 

 What constitutes a successful proof of a mathematical claim?
    The question has challenged philosophers for millennia, but a
    rough and ready definition could be this: A proof of a
    mathematical proposition \coqdocvar{P} is a written (or spoken) text that
    instills in the reader or hearer the certainty that \coqdocvar{P} is true --
    an unassailable argument for the truth of \coqdocvar{P}.  That is, a proof
    is an act of communication.


    Acts of communication may involve different sorts of readers.  On
    one hand, the ''reader`` can be a program like Coq, in which case
    the ''belief`` that is instilled is that \coqdocvar{P} can be mechanically
    derived from a certain set of formal logical rules, and the proof
    is a recipe that guides the program in checking this fact.  Such
    recipes are \textit{formal} proofs.


    Alternatively, the reader can be a human being, in which case the
    proof will be written in English or some other natural language,
    and will thus necessarily be \textit{informal}.  Here, the criteria for
    success are less clearly specified.  A ''valid`` proof is one that
    makes the reader believe \coqdocvar{P}.  But the same proof may be read by
    many different readers, some of whom may be convinced by a
    particular way of phrasing the argument, while others may not be.
    Some readers may be particularly pedantic, inexperienced, or just
    plain thick-headed; the only way to convince them will be to make
    the argument in painstaking detail.  But other readers, more
    familiar in the area, may find all this detail so overwhelming
    that they lose the overall thread; all they want is to be told the
    main ideas, since it is easier for them to fill in the details for
    themselves than to wade through a written presentation of them.
    Ultimately, there is no universal standard, because there is no
    single way of writing an informal proof that is guaranteed to
    convince every conceivable reader.


    In practice, however, mathematicians have developed a rich set of
    conventions and idioms for writing about complex mathematical
    objects that -- at least within a certain community -- make
    communication fairly reliable.  The conventions of this stylized
    form of communication give a fairly clear standard for judging
    proofs good or bad.


    Because we are using Coq in this course, we will be working
    heavily with formal proofs.  But this doesn't mean we can
    completely forget about informal ones!  Formal proofs are useful
    in many ways, but they are \textit{not} very efficient ways of
    communicating ideas between human beings. 

 For example, here is a proof that addition is associative: \begin{coqdoccode}
\coqdocemptyline
\coqdocnoindent
\coqdockw{Theorem} \coqdocvar{plus\_assoc'} : \coqdockw{\ensuremath{\forall}} \coqdocvar{n} \coqdocvar{m} \coqdocvar{p} : \coqdocvar{nat},\coqdoceol
\coqdocindent{1.00em}
\coqdocvar{n} + (\coqdocvar{m} + \coqdocvar{p}) = (\coqdocvar{n} + \coqdocvar{m}) + \coqdocvar{p}.\coqdoceol
 \coqdocemptyline
\end{coqdoccode}
Coq is perfectly happy with this.  For a human, however, it
    is difficult to make much sense of it.  We can use comments and
    bullets to show the structure a little more clearly... \begin{coqdoccode}
\coqdocemptyline
\coqdocnoindent
\coqdockw{Theorem} \coqdocvar{plus\_assoc'{}'} : \coqdockw{\ensuremath{\forall}} \coqdocvar{n} \coqdocvar{m} \coqdocvar{p} : \coqdocvar{nat},\coqdoceol
\coqdocindent{1.00em}
\coqdocvar{n} + (\coqdocvar{m} + \coqdocvar{p}) = (\coqdocvar{n} + \coqdocvar{m}) + \coqdocvar{p}.\coqdoceol
\coqdocemptyline
\end{coqdoccode}
... and if you're used to Coq you may be able to step
    through the tactics one after the other in your mind and imagine
    the state of the context and goal stack at each point, but if the
    proof were even a little bit more complicated this would be next
    to impossible.


    A (pedantic) mathematician might write the proof something like
    this: 


\begin{itemize}
\item  \textit{Theorem}: For any \coqdocvar{n}, \coqdocvar{m} and \coqdocvar{p},


      n + (m + p) = (n + m) + p.


    \textit{Proof}: By induction on \coqdocvar{n}.



\begin{itemize}
\item  First, suppose \coqdocvar{n} = 0.  We must show


        0 + (m + p) = (0 + m) + p.


      This follows directly from the definition of +.



\item  Next, suppose \coqdocvar{n} = \coqdocvar{S} \coqdocvar{n'}, where


        n' + (m + p) = (n' + m) + p.


      We must show


        (S n') + (m + p) = ((S n') + m) + p.


      By the definition of +, this follows from


        S (n' + (m + p)) = S ((n' + m) + p),


      which is immediate from the induction hypothesis.  \textit{Qed}. 
\end{itemize}

\end{itemize}


 The overall form of the proof is basically similar, and of
    course this is no accident: Coq has been designed so that its
    \coqdoctac{induction} tactic generates the same sub-goals, in the same
    order, as the bullet points that a mathematician would write.  But
    there are significant differences of detail: the formal proof is
    much more explicit in some ways (e.g., the use of \coqdoctac{reflexivity})
    but much less explicit in others (in particular, the ''proof state``
    at any given point in the Coq proof is completely implicit,
    whereas the informal proof reminds the reader several times where
    things stand). 

\paragraph{Exercise: 2 stars, advanced, recommended (plus\_comm\_informal)}

 Translate your solution for \coqdocvar{plus\_comm} into an informal proof:


    Theorem: Addition is commutative.


    Proof: \begin{coqdoccode}
\coqdocemptyline
\coqdocnoindent
\coqdockw{Definition} \coqdocvar{manual\_grade\_for\_plus\_comm\_informal} : \coqdocvar{option} (\coqdocvar{prod} \coqdocvar{nat} \coqdocvar{string}) := \coqdocvar{None}.\coqdoceol
\end{coqdoccode}
\ensuremath{\Box} 

\paragraph{Exercise: 2 stars, optional (beq\_nat\_refl\_informal)}

 Write an informal proof of the following theorem, using the
    informal proof of \coqdocvar{plus\_assoc} as a model.  Don't just
    paraphrase the Coq tactics into English!


    Theorem: \coqdocvar{true} = \coqdocvar{beq\_nat} \coqdocvar{n} \coqdocvar{n} for any \coqdocvar{n}.


    Proof:  \ensuremath{\Box} \begin{coqdoccode}
\coqdocemptyline
\end{coqdoccode}
\section{More Exercises}



\paragraph{Exercise: 3 stars, recommended (mult\_comm)}

 Use \coqdoctac{assert} to help prove this theorem.  You shouldn't need to
    use induction on \coqdocvar{plus\_swap}. \begin{coqdoccode}
\coqdocemptyline
\coqdocnoindent
\coqdockw{Theorem} \coqdocvar{plus\_swap} : \coqdockw{\ensuremath{\forall}} \coqdocvar{n} \coqdocvar{m} \coqdocvar{p} : \coqdocvar{nat},\coqdoceol
\coqdocindent{1.00em}
\coqdocvar{n} + (\coqdocvar{m} + \coqdocvar{p}) = \coqdocvar{m} + (\coqdocvar{n} + \coqdocvar{p}).\coqdoceol
 \coqdocemptyline
\end{coqdoccode}
Now prove commutativity of multiplication.  (You will probably
    need to define and prove a separate subsidiary theorem to be used
    in the proof of this one.  You may find that \coqdocvar{plus\_swap} comes in
    handy.) \begin{coqdoccode}
\coqdocemptyline
\coqdocnoindent
\coqdockw{Theorem} \coqdocvar{mult\_comm} : \coqdockw{\ensuremath{\forall}} \coqdocvar{m} \coqdocvar{n} : \coqdocvar{nat},\coqdoceol
\coqdocindent{1.00em}
\coqdocvar{m} \ensuremath{\times} \coqdocvar{n} = \coqdocvar{n} \ensuremath{\times} \coqdocvar{m}.\coqdoceol
 \end{coqdoccode}
\ensuremath{\Box} 

\paragraph{Exercise: 3 stars, optional (more\_exercises)}

 Take a piece of paper.  For each of the following theorems, first
    \textit{think} about whether (a) it can be proved using only
    simplification and rewriting, (b) it also requires case
    analysis (\coqdoctac{destruct}), or (c) it also requires induction.  Write
    down your prediction.  Then fill in the proof.  (There is no need
    to turn in your piece of paper; this is just to encourage you to
    reflect before you hack!) \begin{coqdoccode}
\coqdocemptyline
\coqdocnoindent
\coqdockw{Check} \coqdocvar{leb}.\coqdoceol
\coqdocemptyline
\coqdocnoindent
\coqdockw{Theorem} \coqdocvar{leb\_refl} : \coqdockw{\ensuremath{\forall}} \coqdocvar{n}:\coqdocvar{nat},\coqdoceol
\coqdocindent{1.00em}
\coqdocvar{true} = \coqdocvar{leb} \coqdocvar{n} \coqdocvar{n}.\coqdoceol
 \coqdocemptyline
\coqdocnoindent
\coqdockw{Theorem} \coqdocvar{zero\_nbeq\_S} : \coqdockw{\ensuremath{\forall}} \coqdocvar{n}:\coqdocvar{nat},\coqdoceol
\coqdocindent{1.00em}
\coqdocvar{beq\_nat} 0 (\coqdocvar{S} \coqdocvar{n}) = \coqdocvar{false}.\coqdoceol
 \coqdocemptyline
\coqdocnoindent
\coqdockw{Theorem} \coqdocvar{andb\_false\_r} : \coqdockw{\ensuremath{\forall}} \coqdocvar{b} : \coqdocvar{bool},\coqdoceol
\coqdocindent{1.00em}
\coqdocvar{andb} \coqdocvar{b} \coqdocvar{false} = \coqdocvar{false}.\coqdoceol
 \coqdocemptyline
\coqdocnoindent
\coqdockw{Theorem} \coqdocvar{plus\_ble\_compat\_l} : \coqdockw{\ensuremath{\forall}} \coqdocvar{n} \coqdocvar{m} \coqdocvar{p} : \coqdocvar{nat},\coqdoceol
\coqdocindent{1.00em}
\coqdocvar{leb} \coqdocvar{n} \coqdocvar{m} = \coqdocvar{true} \ensuremath{\rightarrow} \coqdocvar{leb} (\coqdocvar{p} + \coqdocvar{n}) (\coqdocvar{p} + \coqdocvar{m}) = \coqdocvar{true}.\coqdoceol
 \coqdocemptyline
\coqdocnoindent
\coqdockw{Theorem} \coqdocvar{S\_nbeq\_0} : \coqdockw{\ensuremath{\forall}} \coqdocvar{n}:\coqdocvar{nat},\coqdoceol
\coqdocindent{1.00em}
\coqdocvar{beq\_nat} (\coqdocvar{S} \coqdocvar{n}) 0 = \coqdocvar{false}.\coqdoceol
 \coqdocemptyline
\coqdocnoindent
\coqdockw{Theorem} \coqdocvar{mult\_1\_l} : \coqdockw{\ensuremath{\forall}} \coqdocvar{n}:\coqdocvar{nat}, 1 \ensuremath{\times} \coqdocvar{n} = \coqdocvar{n}.\coqdoceol
 \coqdocemptyline
\coqdocnoindent
\coqdockw{Theorem} \coqdocvar{all3\_spec} : \coqdockw{\ensuremath{\forall}} \coqdocvar{b} \coqdocvar{c} : \coqdocvar{bool},\coqdoceol
\coqdocindent{2.00em}
\coqdocvar{orb}\coqdoceol
\coqdocindent{3.00em}
(\coqdocvar{andb} \coqdocvar{b} \coqdocvar{c})\coqdoceol
\coqdocindent{3.00em}
(\coqdocvar{orb} (\coqdocvar{negb} \coqdocvar{b})\coqdoceol
\coqdocindent{7.50em}
(\coqdocvar{negb} \coqdocvar{c}))\coqdoceol
\coqdocindent{1.00em}
= \coqdocvar{true}.\coqdoceol
 \coqdocemptyline
\coqdocnoindent
\coqdockw{Theorem} \coqdocvar{mult\_plus\_distr\_r} : \coqdockw{\ensuremath{\forall}} \coqdocvar{n} \coqdocvar{m} \coqdocvar{p} : \coqdocvar{nat},\coqdoceol
\coqdocindent{1.00em}
(\coqdocvar{n} + \coqdocvar{m}) \ensuremath{\times} \coqdocvar{p} = (\coqdocvar{n} \ensuremath{\times} \coqdocvar{p}) + (\coqdocvar{m} \ensuremath{\times} \coqdocvar{p}).\coqdoceol
 \coqdocemptyline
\coqdocnoindent
\coqdockw{Theorem} \coqdocvar{mult\_assoc} : \coqdockw{\ensuremath{\forall}} \coqdocvar{n} \coqdocvar{m} \coqdocvar{p} : \coqdocvar{nat},\coqdoceol
\coqdocindent{1.00em}
\coqdocvar{n} \ensuremath{\times} (\coqdocvar{m} \ensuremath{\times} \coqdocvar{p}) = (\coqdocvar{n} \ensuremath{\times} \coqdocvar{m}) \ensuremath{\times} \coqdocvar{p}.\coqdoceol
 \end{coqdoccode}
\ensuremath{\Box} 

\paragraph{Exercise: 2 stars, optional (beq\_nat\_refl)}

 Prove the following theorem.  (Putting the \coqdocvar{true} on the left-hand
    side of the equality may look odd, but this is how the theorem is
    stated in the Coq standard library, so we follow suit.  Rewriting
    works equally well in either direction, so we will have no problem
    using the theorem no matter which way we state it.) \begin{coqdoccode}
\coqdocemptyline
\coqdocnoindent
\coqdockw{Theorem} \coqdocvar{beq\_nat\_refl} : \coqdockw{\ensuremath{\forall}} \coqdocvar{n} : \coqdocvar{nat},\coqdoceol
\coqdocindent{1.00em}
\coqdocvar{true} = \coqdocvar{beq\_nat} \coqdocvar{n} \coqdocvar{n}.\coqdoceol
 \end{coqdoccode}
\ensuremath{\Box} 

\paragraph{Exercise: 2 stars, optional (plus\_swap')}

 The \coqdoctac{replace} tactic allows you to specify a particular subterm to
   rewrite and what you want it rewritten to: \coqdoctac{replace} (\coqdocvar{t}) \coqdockw{with} (\coqdocvar{u})
   replaces (all copies of) expression \coqdocvar{t} in the goal by expression
   \coqdocvar{u}, and generates \coqdocvar{t} = \coqdocvar{u} as an additional subgoal. This is often
   useful when a plain \coqdoctac{rewrite} acts on the wrong part of the goal.


   Use the \coqdoctac{replace} tactic to do a proof of \coqdocvar{plus\_swap'}, just like
   \coqdocvar{plus\_swap} but without needing \coqdoctac{assert} (\coqdocvar{n} + \coqdocvar{m} = \coqdocvar{m} + \coqdocvar{n}). \begin{coqdoccode}
\coqdocemptyline
\coqdocnoindent
\coqdockw{Theorem} \coqdocvar{plus\_swap'} : \coqdockw{\ensuremath{\forall}} \coqdocvar{n} \coqdocvar{m} \coqdocvar{p} : \coqdocvar{nat},\coqdoceol
\coqdocindent{1.00em}
\coqdocvar{n} + (\coqdocvar{m} + \coqdocvar{p}) = \coqdocvar{m} + (\coqdocvar{n} + \coqdocvar{p}).\coqdoceol
 \end{coqdoccode}
\ensuremath{\Box} 

\paragraph{Exercise: 3 stars, recommended (binary\_commute)}

 Recall the \coqdocvar{incr} and \coqdocvar{bin\_to\_nat} functions that you
    wrote for the \coqdocvar{binary} exercise in the \coqdocvar{Basics} chapter.  Prove
    that the following diagram commutes:


                            incr
              bin ----------------------> bin
               |                           |
    bin\_to\_nat |                           |  bin\_to\_nat
               |                           |
               v                           v
              nat ----------------------> nat
                             S


    That is, incrementing a binary number and then converting it to
    a (unary) natural number yields the same result as first converting
    it to a natural number and then incrementing.
    Name your theorem \coqdocvar{bin\_to\_nat\_pres\_incr} (''pres`` for ''preserves``).


    Before you start working on this exercise, copy the definitions
    from your solution to the \coqdocvar{binary} exercise here so that this file
    can be graded on its own.  If you want to change your original
    definitions to make the property easier to prove, feel free to
    do so! \begin{coqdoccode}
\coqdocemptyline
\coqdocemptyline
\coqdocnoindent
\coqdockw{Definition} \coqdocvar{manual\_grade\_for\_binary\_commute} : \coqdocvar{option} (\coqdocvar{prod} \coqdocvar{nat} \coqdocvar{string}) := \coqdocvar{None}.\coqdoceol
\end{coqdoccode}
\ensuremath{\Box} 

\paragraph{Exercise: 5 stars, advanced (binary\_inverse)}

 This exercise is a continuation of the previous exercise about
    binary numbers.  You will need your definitions and theorems from
    there to complete this one; please copy them to this file to make
    it self contained for grading.


    (a) First, write a function to convert natural numbers to binary
        numbers.  Then prove that starting with any natural number,
        converting to binary, then converting back yields the same
        natural number you started with.


    (b) You might naturally think that we should also prove the
        opposite direction: that starting with a binary number,
        converting to a natural, and then back to binary yields the
        same number we started with.  However, this is not true!
        Explain what the problem is.


    (c) Define a ''direct`` normalization function -- i.e., a function
        \coqdocvar{normalize} from binary numbers to binary numbers such that,
        for any binary number b, converting to a natural and then back
        to binary yields (\coqdocvar{normalize} \coqdocvar{b}).  Prove it.  (Warning: This
        part is tricky!)


    Again, feel free to change your earlier definitions if this helps
    here. \begin{coqdoccode}
\coqdocemptyline
\coqdocemptyline
\coqdocnoindent
\coqdockw{Definition} \coqdocvar{manual\_grade\_for\_binary\_inverse} : \coqdocvar{option} (\coqdocvar{prod} \coqdocvar{nat} \coqdocvar{string}) := \coqdocvar{None}.\coqdoceol
\end{coqdoccode}
\ensuremath{\Box} \begin{coqdoccode}
\coqdocemptyline
\end{coqdoccode}
\end{document}
